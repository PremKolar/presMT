%\newcommand{\HRule}{\rule{\linewidth}{0.5mm}} % Defines a new command for the horizontal lines, change thickness here
%\newcommand*{\eg}{e.g.\@\xspace}
%\newcommand*{\ie}{i.e.\@\xspace}
%\newcommand*{\Eg}{E.g.\@\xspace}
%\newcommand*{\Ie}{I.e.\@\xspace}
%\makeatletter
%\newcommand*{\etc}{%
    %\@ifnextchar{.}%
        %{etc}%
        %{etc.\@\xspace}%
%}
%\let\oldhat\hat

%\newcommand{\rnm}[1]{\romannumeral #1}
%\newcommand{\Rmn}[1]{\Roman #1}
\renewcommand{\exp}[1]{\mathrm{e}^{#1}}
\renewcommand{\vec}[1]{\mathbf{\bm{#1}}}
%\newcommand{\ten}[1]{\mathbb{\bm{#1}}}
%\newcommand{\tent}[1]{\mathbb{\bm{#1}}^{\mathsmaller T}}
%\renewcommand{\hat}[1]{\oldhat{\mathbf{#1}}}
\renewcommand{\l}{\left(}
 \renewcommand{\r}{\right)}
\newcommand{\pr}{\partial}
\newcommand{\grad}{\vec{\nabla}}
%\newcommand{\arccosh}{\mathrm{arccosh}}
%\newcommand{\lapl}{\vec{\triangle}}
\newcommand{\curl}{\grad \times}
\newcommand{\vort}[1]{\grad \times \vec{#1}}
\newcommand{\gradt}{ \breve{\grad}}
\renewcommand{\div}{\grad \cdot}
\newcommand{\norm}[1]{\left\lVert#1\right\rVert}
%\newcommand{\footder}[1]{\footnote{See section \ref{#1} for derivation.}}
%\let\oldeqref\eqref
%\renewcommand{\eqref}[1]{equation \oldeqref{#1}}
%\newcommand{\eqsref}[1]{equations \oldeqref{#1}}
%\newcommand{\Eqref}[1]{Equation \oldeqref{#1}}
%\newcommand{\Eqsref}[1]{Equations \oldeqref{#1}}
%\newcommand{\dpr}[2]{\frac{\partial#1}{\partial#2}}
%\newcommand{\Dpr}[2]{\frac{D#1}{D#2}}
%\newcommand{\Dprs}[2]{\frac{D^{\star}#1}{D#2}}
%\newcommand{\advec}[1]{\vec{u} \cdot \vec{\nabla} #1}
%\newcommand{\unit}[1]{\left[#1\right]}
%\newcommand{\unitvec}[1]{\hat{\vec{#1}}}
\newcommand{\tvec}[1]{\underaccent{\neg}{\vec{#1}}}
\newcommand{\tsca}[1]{\underaccent{\neg}{#1}}
\newcommand{\reflvec}[1]{\underaccent{\leftarrow}{\vec{#1}}}
%\newcommand{\order}[1]{\mathcal{O}\left( 10^{#1}\right)}
%\newcommand{\sign}[1]{\mathrm{sgn}\left(#1\right)}
%\newenvironment{colbox}[1]
%{
	%\def\FrameCommand{\colorbox{colboxcolor}}%
	%\MakeFramed{\advance\hsize-\width \FrameRestore\cornersize{0.9}}
	%\begin{scriptsize}
	%\section{#1}
%}
%{
      %\end{scriptsize}
	%\endMakeFramed
%}

%\newenvironment{colbox2}[1]
%{

	%\def\FrameCommand{\colorbox{legendcol}}%
	%\MakeFramed{\advance\hsize-\width \FrameRestore}
	%\begin{footnotesize}
%}
%{
%\end{footnotesize}
	%\endMakeFramed
%}

 %\definecolor{colboxcolor}{HTML}{9ECBFF}
%\definecolor{legendcol}{HTML}{7BBCBD}

%%%####################################################################
%%%####################################################################
%\newenvironment{kingbox}[1]
%{
	%%\begin{minipage}{.4\textwidth}
	%\begin{center}
	%\begin{framed}
%%\begin{description}
%\textbf{#1} \\
%}
%{
%%\end{description}
	%\end{framed}
	%\end{center}
	%%\end{minipage}
	%}



%%%####################################################################
%%%####################################################################
%\DeclareMathOperator{\tr}{Tr}
%\newcommand*{\supervisor}[1]{\def\supname{#1}}
%\newcommand*{\thesistitle}[1]{\def\ttitle{#1}}
%\newcommand*{\examiner}[1]{\def\examname{#1}}
%\newcommand*{\degree}[1]{\def\degreename{#1}}
%\newcommand*{\authors}[1]{\def\authornames{#1}}
%\newcommand*{\addresses}[1]{\def\addressnames{#1}}
%\newcommand*{\university}[1]{\def\univname{#1}}
%\newcommand*{\UNIVERSITY}[1]{\def\UNIVNAME{#1}}
%\newcommand*{\department}[1]{\def\deptname{#1}}
%\newcommand*{\DEPARTMENT}[1]{\def\DEPTNAME{#1}}
%\newcommand*{\group}[1]{\def\groupname{#1}}
%\newcommand*{\GROUP}[1]{\def\GROUPNAME{#1}}
%\newcommand*{\faculty}[1]{\def\facname{#1}}
%\newcommand*{\FACULTY}[1]{\def\FACNAME{#1}}
%\newcommand\btypeout[1]{\bhrule\typeout{\space #1}\bhrule}
%\newcommand\bhrule{\typeout{------------------------------------------------------------------------------}}
%\newcommand\Declaration[1]{
%\btypeout{Declaration of Authorship}
%\thispagestyle{plain}
%\null\vfil
%%\vskip 60\p@
%\begin{center}{\huge\bf Declaration of Authorship\par}\end{center}
%%\vskip 60\p@
%{\normalsize #1}
%\vfil\vfil\null
%%\cleardoublepage
%}


%\newcommand{\Bu}[0]{\mathrm{\hyperref[def:Bu]{Bu}}\;}
%\newcommand{\Ro}[0]{\mathrm{\hyperref[def:Ro]{Ro}}\;}
%\newcommand{\Rh}[0]{\mathrm{\hyperref[def:Rh]{R_{\beta}}}\;}
%\newcommand{\Lr}[0]{\mathrm{\hyperref[def:Lr]{L_{R}}}\;}
%\newcommand{\Lb}[0]{\mathrm{\hyperref[def:Lb]{L_{\beta}}}\;}
%\newcommand{\h}[0]{\hyperref[def:h]{h}}
%\newcommand{\B}[0]{\hyperref[def:B]{ \vec{B}}}
%\newcommand{\Ek}[0]{\hyperref[def:Ek]{E_k}}
%\newcommand{\Em}[0]{\hyperref[def:Em]{E_m}}
%~ \newcommand{\enstro}[0]{\hyperref[def:enstro]{\varepsilon}}
\newcommand{\enstro}{\varepsilon}
%\newcommand{\f}[0]{\mathit{\hyperref[def:f]{f}}\;}
%\newcommand{\dfdy}[0]{\mathrm{\hyperref[def:beta]{\beta}}\;}
%\newcommand{\g}[0]{\mathit{\hyperref[def:g]{g}}\;}
%\newcommand{\Enstro}{\mathcal{E}}
%\newcommand{\okubo}[0]{\mathrm{\hyperref[def:okubo]{O_w}}\;}
%\newcommand{\SSH}[0]{\hyperref[def:SSH]{SSH}\;}
%\newcommand{\IQ}[0]{\mathrm{\hyperref[def:IQ]{IQ}}\;}
%\newcommand{\inta}[1]{ \int_A #1 \; \mathrm{d}A}
%\newcommand{\intm}[1]{ \frac{1}{A} \int_A #1 \; \mathrm{d}A}
%\newcommand{\dint}[1]{ \; \mathrm{d}#1}
%\newcommand{\intms}[1]{ \left< #1 \right>}
%\newcommand{\rG}[0]{\hyperref[def:rG]{\mathfrak{r}}\;}
%\makeatletter
%\newcommand{\litem}[2]{%
%\def\@itemlabel{\textbf{#1}}
%\item
%\def\@currentlabel{\textbf{#1}}\label{#2}}
%\makeatother
%\newcommand{\decom}[1]{\overline{#1} + #1'}
%\newcommand{\inbr}[1]{\left( #1 \right)}
%\newcommand{\ol}[1]{\overline{#1}}
%\newcommand{\timesES}[2]{\epsilon_{jki} #1_j #2_k \unitvec{e}_i}
%\newcommand{\oh}[0]{\frac{1}{2}}
%\newcommand{\todoil}[1]{\todo[color=red]{#1}}
%\newcommand{\todoilc}[2]{\todo[color=#1]{#2}}
%\newcommand{\derref}[1]{\footnote{see\fcolorbox{white}{gray!20}{Derivation \ref{#1}}}
%}
%\newcommand{\derrefs}[1]{\footnote{see\fcolorbox{white}{gray!20}{Derivations \cref{#1}}}
%}
%% \mdfdefinestyle{definition}
%% {linewidth=1,
%% backgroundcolor=yellow!40,
%% outerlinecolor=blue!70!black,
%% frametitlebackgroundcolor=gray!20,
%% % frametitlerule=true,
%% innertopmargin=\topskip,}
%% \mdtheorem[
%% style=definition]{definition}{Definition}
%% \mdfdefinestyle{codepiece}
%% {linewidth=15pt,
%% linecolor=black,
%% %frametitlerule=true,
%% backgroundcolor=red!1,
%% outerlinecolor=black,
%% frametitlebackgroundcolor=blue!10,
%% innertopmargin=\topskip,}
%% \mdtheorem[
%% style=codepiece]{codepiece}{Code}[chapter]
%% \mdfdefinestyle{function}
%% {%linewidth=15pt,
%% %linecolor=black,
%% %frametitlerule=true,
%% %backgroundcolor=red!1,
%% %outerlinecolor=black,
%% %frametitlebackgroundcolor=blue!10,
%% innertopmargin=-.5cm,}
%% \mdtheorem[
%% style=function]{function}{sub-routine}[section]
%% \mdfdefinestyle{derivation}
%% {linecolor=red,
%% outerlinewidth=2,
%% leftmargin=40mm,
%% rightmargin=40mm,
%% linewidth=2,
%% frametitlerule=true,
%% frametitlebackgroundcolor=gray!20,
%% innertopmargin=\topskip,roundcorner=10pt,}
%% \mdtheorem[
%% style=derivation]{derivation}{Derivation}
%% \mdfdefinestyle{eddy}
%% {linewidth=.5,
%% % align=center,
%% % backgroundcolor=yellow!40,
%% outerlinecolor=black,
%% frametitlebackgroundcolor=gray!20,
%% % frametitlerule=true,
%% innertopmargin=\topskip,}
%% \mdtheorem[
%% style=eddy]{eddy}{Vortex}[chapter]
%% \mdtheorem[
%% style=eddy]{turbu}{Turbulence}[chapter]




\newlength\dlf
\newcommand\alignedbox[3][yellow]{
  % #1 = color (optional, defaults to yellow)
  % #2 = before alignment
  % #3 = after alignment
  &
  \begingroup
  \settowidth\dlf{$\displaystyle #2$}
  \addtolength\dlf{\fboxsep+\fboxrule}
  \hspace{-\dlf}
  \fcolorbox{red}{#1}{$\displaystyle #2 #3$}
  \endgroup
}



