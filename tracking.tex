%effectively looking at zero om contours. fickle (similar to OW, which is even d_xx squared!). gladly pop and aviso product are rel. clean in terms of noise.
% note: dyn radius <=> std(gaussbell)
% f(x)=exp(- (2 r_dyn)^-2   x^2 )
\begin{frame}
 \frametitle{$ \partial_x \sim \vec{U}, \; \partial_{xx} \sim \vec{\omega}, \;  \left(\partial_{xx}\right)^2 \sim OW \sim \enstro$}
\begin{figure}
	\centering
% 	\includegraphics[width=300pt,keepaspectratio=true]{GS.pdf}
	\includegraphics[height=220pt,keepaspectratio=true]{prof1.pdf}
\end{figure}
\end{frame}



\section{Tracking}
\begin{frame}
 \frametitle{$ \partial_x \sim \vec{U}, \; \partial_{xx} \sim \vec{\omega}, \;  \left(\partial_{xx}\right)^2 \sim OW \sim \enstro$}
\begin{figure}
	\centering
% 	\includegraphics[width=300pt,keepaspectratio=true]{GS.pdf}
	\includegraphics[height=220pt,keepaspectratio=true]{prof2.pdf}
\end{figure}
\end{frame}

\section{Tracking}
\begin{frame}
 \frametitle{$ y(x)=\sqrt{e} a \exp{\left(- x^2/2\sigma^2\right)}$}
\begin{figure}
	\centering
% 	\includegraphics[width=300pt,keepaspectratio=true]{GS.pdf}
	\includegraphics[height=220pt,keepaspectratio=true]{prof3.pdf}
\end{figure}
\end{frame}


\section{Tracking}
\begin{frame}
 \frametitle{$ y(x)=\sqrt{e} a \exp{\left(- x^2/2\sigma^2\right)}$}
\begin{figure}
	\centering
% 	\includegraphics[width=300pt,keepaspectratio=true]{GS.pdf}
	\includegraphics[height=220pt,keepaspectratio=true]{prof3b.pdf}
\end{figure}
\end{frame}

\section{Tracking}
\begin{frame}
 %\frametitle{$ y(x)=\sqrt{e} a \exp{\left(- x^2/2\sigma^2\right)}$}
\begin{figure}
	%\centering
% 	\includegraphics[width=300pt,keepaspectratio=true]{GS.pdf}
	\includegraphics[width=300pt,keepaspectratio=true]{prof3bPsi.pdf}
\end{figure}
\end{frame}

\begin{frame}
% NOTE: 14day dt
% \frametitle{pictures in latex beamer class}
\begin{figure}
	\centering
% 	\includegraphics[width=300pt,keepaspectratio=true]{GS.pdf}
	\includegraphics[height=250pt,keepaspectratio=true]{shrunks/scatRadOLrComp.pdf}
\end{figure}
\end{frame}




\begin{frame}
% \frametitle{pictures in latex beamer class}
\begin{figure}
	\centering
% 	\includegraphics[width=300pt,keepaspectratio=true]{GS.pdf}
	\includegraphics[width=300pt,keepaspectratio=true]{shrunks/parAnalyCH.pdf}
\end{figure}
\end{frame}




\begin{frame}
% \frametitle{pictures in latex beamer class}
\begin{figure}
	\centering
% 	\includegraphics[width=300pt,keepaspectratio=true]{GS.pdf}
	\includegraphics[width=300pt,keepaspectratio=true]{shrunks/parAnalyIQ2.pdf}
	%\includegraphics[width=300pt,keepaspectratio=true]{shrunks/parAnalyIQ2.pdf}
\end{figure}
\end{frame}



\begin{frame}
% \frametitle{pictures in latex beamer class}
\begin{figure}
	\centering
% 	\includegraphics[width=300pt,keepaspectratio=true]{GS.pdf}
	\includegraphics[width=300pt,keepaspectratio=true]{shrunks/parAnalyIQ6.pdf}
\end{figure}
\end{frame}

\begin{frame}
% \frametitle{pictures in latex beamer class}
\begin{figure}
	\centering
% 	\includegraphics[width=300pt,keepaspectratio=true]{GS.pdf}
	\includegraphics[width=300pt,keepaspectratio=true]{crpd_scat1_shrunk2prepress.pdf}
\end{figure}
\end{frame}
\begin{frame}
% \frametitle{pictures in latex beamer class}
\begin{figure}
	\centering
% 	\includegraphics[width=300pt,keepaspectratio=true]{GS.pdf}
	\includegraphics[width=300pt,keepaspectratio=true]{deflectsCycs_shrunk2prepress.pdf}
\end{figure}
\end{frame}



\begin{frame}
 \frametitle{Isoperimetric Quotient (IQ)}
\begin{figure}
	\centering
% 	\includegraphics[width=300pt,keepaspectratio=true]{GS.pdf}
	\includegraphics[height=220pt,keepaspectratio=true]{MapVelCycs.pdf}
\end{figure}
\end{frame}

\begin{frame}
 \frametitle{Isoperimetric Quotient (IQ)}
\begin{figure}
% 	\includegraphics[width=300pt,keepaspectratio=true]{GS.pdf}
	\includegraphics[height=150pt,keepaspectratio=true]{shrunks/isoper.pdf}
\end{figure}
\end{frame}


