\documentclass{beamer}
\setbeamertemplate{footline}[infolines theme]
% \usepackage{beamerthemeshadow}

\usepackage{wrapfig}
\usepackage{xcolor,colortbl}
%\usepackage{bibtex}
\usepackage[numbers]{natbib}
%\usepackage{bibentry}
%\usepackage{chngcntr}
%\usepackage{multimedia}
% \usepackage[english]{babel}
% \usepackage[utf8]{inputenc}
% \usepackage{cleveref}
% \usepackage{colortbl}
% \usepackage{color}
% \usepackage{tikz}
% \usepackage{fancyhdr}
% \usepackage{url}
% \usepackage{setspace}
% \usepackage{framed}
% \usepackage{listings}
% \usepackage{wrapfig}
% \usepackage{enumerate}
% \usepackage{bm}
% \usepackage{cancel}
% \usepackage{graphics}
% \usepackage{amsmath,graphicx}
% \usepackage{amsfonts}
% \usepackage{amssymb}
% \usepackage{calc}
% \usepackage{accents}
% \usepackage{xspace}
% \usepackage{setspace}
% \usepackage{lipsum}
% %\usepackage{vmargin}
% \usepackage{multicol}
% %\usepackage{fullpage}
% \usepackage{nameref}
% %\usepackage[toc]{multitoc}
% \usepackage{enumitem}
% \usepackage[sort]{natbib}
% \usepackage[colorinlistoftodos]{todonotes}
% \setlength{\marginparwidth}{3cm}
% \usepackage{fancybox}
% \usepackage{tabularx}
% \usetikzlibrary{shapes,arrows}
% \raggedbottom
% \usepackage{wrapfig}
% \usepackage{geometry}
% % \usepackage[framemethod=TikZ]{mdframed}
% \usepackage[titletoc]{appendix}
% % \usepackage{kvoptions, xparse, etoolbox, color, xcolor, tikz, pstricks}
% \usepackage{kvoptions, etoolbox, color, xcolor, tikz, pstricks}
% \usepackage{hyperref}
% \definecolor{linkcolor}{HTML}{00236D}
% \hypersetup{colorlinks, citecolor=blue, filecolor=blue, linkcolor=linkcolor,
% urlcolor=blue}
% \usetikzlibrary{shapes,arrows}
% \usepackage{verbatim}
% \usepackage{float}

%\newcommand{\HRule}{\rule{\linewidth}{0.5mm}} % Defines a new command for the horizontal lines, change thickness here
%\newcommand*{\eg}{e.g.\@\xspace}
%\newcommand*{\ie}{i.e.\@\xspace}
%\newcommand*{\Eg}{E.g.\@\xspace}
%\newcommand*{\Ie}{I.e.\@\xspace}
%\makeatletter
%\newcommand*{\etc}{%
    %\@ifnextchar{.}%
        %{etc}%
        %{etc.\@\xspace}%
%}
%\let\oldhat\hat

%\newcommand{\rnm}[1]{\romannumeral #1}
%\newcommand{\Rmn}[1]{\Roman #1}
%\renewcommand{\vec}[1]{\mathbf{\bm{#1}}}
\renewcommand{\vec}[1]{\mathbf{#1}}
%\newcommand{\ten}[1]{\mathbb{\bm{#1}}}
%\newcommand{\tent}[1]{\mathbb{\bm{#1}}^{\mathsmaller T}}
%\renewcommand{\hat}[1]{\oldhat{\mathbf{#1}}}
%\renewcommand{\l}{\left(}
 %\renewcommand{\r}{\right)}
%\newcommand{\pr}{\partial}
\newcommand{\grad}{\vec{\nabla}}
%\newcommand{\arccosh}{\mathrm{arccosh}}
%\newcommand{\lapl}{\vec{\triangle}}
%\newcommand{\curl}{\grad \times}
%\newcommand{\gradt}{ \breve{\grad}}
%\renewcommand{\div}{\grad \cdot}
%\newcommand{\norm}[1]{\left\lVert#1\right\rVert}
%\newcommand{\footder}[1]{\footnote{See section \ref{#1} for derivation.}}
%\let\oldeqref\eqref
%\renewcommand{\eqref}[1]{equation \oldeqref{#1}}
%\newcommand{\eqsref}[1]{equations \oldeqref{#1}}
%\newcommand{\Eqref}[1]{Equation \oldeqref{#1}}
%\newcommand{\Eqsref}[1]{Equations \oldeqref{#1}}
%\newcommand{\dpr}[2]{\frac{\partial#1}{\partial#2}}
%\newcommand{\Dpr}[2]{\frac{D#1}{D#2}}
%\newcommand{\Dprs}[2]{\frac{D^{\star}#1}{D#2}}
%\newcommand{\advec}[1]{\vec{u} \cdot \vec{\nabla} #1}
%\newcommand{\unit}[1]{\left[#1\right]}
%\newcommand{\unitvec}[1]{\hat{\vec{#1}}}
%\newcommand{\tvec}[1]{\underaccent{\neg}{\vec{#1}}}
%\newcommand{\tsca}[1]{\underaccent{\neg}{#1}}
%\newcommand{\order}[1]{\mathcal{O}\left( 10^{#1}\right)}
%\newcommand{\sign}[1]{\mathrm{sgn}\left(#1\right)}
%\newenvironment{colbox}[1]
%{
	%\def\FrameCommand{\colorbox{colboxcolor}}%
	%\MakeFramed{\advance\hsize-\width \FrameRestore\cornersize{0.9}}
	%\begin{scriptsize}
	%\section{#1}
%}
%{
      %\end{scriptsize}
	%\endMakeFramed
%}

%\newenvironment{colbox2}[1]
%{

	%\def\FrameCommand{\colorbox{legendcol}}%
	%\MakeFramed{\advance\hsize-\width \FrameRestore}
	%\begin{footnotesize}
%}
%{
%\end{footnotesize}
	%\endMakeFramed
%}

 %\definecolor{colboxcolor}{HTML}{9ECBFF}
%\definecolor{legendcol}{HTML}{7BBCBD}

%%%####################################################################
%%%####################################################################
%\newenvironment{kingbox}[1]
%{
	%%\begin{minipage}{.4\textwidth}
	%\begin{center}
	%\begin{framed}
%%\begin{description}
%\textbf{#1} \\
%}
%{
%%\end{description}
	%\end{framed}
	%\end{center}
	%%\end{minipage}
	%}



%%%####################################################################
%%%####################################################################
%\DeclareMathOperator{\tr}{Tr}
%\newcommand*{\supervisor}[1]{\def\supname{#1}}
%\newcommand*{\thesistitle}[1]{\def\ttitle{#1}}
%\newcommand*{\examiner}[1]{\def\examname{#1}}
%\newcommand*{\degree}[1]{\def\degreename{#1}}
%\newcommand*{\authors}[1]{\def\authornames{#1}}
%\newcommand*{\addresses}[1]{\def\addressnames{#1}}
%\newcommand*{\university}[1]{\def\univname{#1}}
%\newcommand*{\UNIVERSITY}[1]{\def\UNIVNAME{#1}}
%\newcommand*{\department}[1]{\def\deptname{#1}}
%\newcommand*{\DEPARTMENT}[1]{\def\DEPTNAME{#1}}
%\newcommand*{\group}[1]{\def\groupname{#1}}
%\newcommand*{\GROUP}[1]{\def\GROUPNAME{#1}}
%\newcommand*{\faculty}[1]{\def\facname{#1}}
%\newcommand*{\FACULTY}[1]{\def\FACNAME{#1}}
%\newcommand\btypeout[1]{\bhrule\typeout{\space #1}\bhrule}
%\newcommand\bhrule{\typeout{------------------------------------------------------------------------------}}
%\newcommand\Declaration[1]{
%\btypeout{Declaration of Authorship}
%\thispagestyle{plain}
%\null\vfil
%%\vskip 60\p@
%\begin{center}{\huge\bf Declaration of Authorship\par}\end{center}
%%\vskip 60\p@
%{\normalsize #1}
%\vfil\vfil\null
%%\cleardoublepage
%}


%\newcommand{\Bu}[0]{\mathrm{\hyperref[def:Bu]{Bu}}\;}
%\newcommand{\Ro}[0]{\mathrm{\hyperref[def:Ro]{Ro}}\;}
%\newcommand{\Rh}[0]{\mathrm{\hyperref[def:Rh]{R_{\beta}}}\;}
%\newcommand{\Lr}[0]{\mathrm{\hyperref[def:Lr]{L_{R}}}\;}
%\newcommand{\Lb}[0]{\mathrm{\hyperref[def:Lb]{L_{\beta}}}\;}
%\newcommand{\h}[0]{\hyperref[def:h]{h}}
%\newcommand{\B}[0]{\hyperref[def:B]{ \vec{B}}}
%\newcommand{\Ek}[0]{\hyperref[def:Ek]{E_k}}
%\newcommand{\Em}[0]{\hyperref[def:Em]{E_m}}
\newcommand{\enstro}[0]{\varepsilon}
%\newcommand{\f}[0]{\mathit{\hyperref[def:f]{f}}\;}
%\newcommand{\dfdy}[0]{\mathrm{\hyperref[def:beta]{\beta}}\;}
%\newcommand{\g}[0]{\mathit{\hyperref[def:g]{g}}\;}
%\newcommand{\Enstro}{\mathcal{E}}
%\newcommand{\okubo}[0]{\mathrm{\hyperref[def:okubo]{O_w}}\;}
%\newcommand{\SSH}[0]{\hyperref[def:SSH]{SSH}\;}
%\newcommand{\IQ}[0]{\mathrm{\hyperref[def:IQ]{IQ}}\;}
%\newcommand{\inta}[1]{ \int_A #1 \; \mathrm{d}A}
%\newcommand{\intm}[1]{ \frac{1}{A} \int_A #1 \; \mathrm{d}A}
%\newcommand{\dint}[1]{ \; \mathrm{d}#1}
%\newcommand{\intms}[1]{ \left< #1 \right>}
%\newcommand{\rG}[0]{\hyperref[def:rG]{\mathfrak{r}}\;}
%\makeatletter
%\newcommand{\litem}[2]{%
%\def\@itemlabel{\textbf{#1}}
%\item
%\def\@currentlabel{\textbf{#1}}\label{#2}}
%\makeatother
%\newcommand{\decom}[1]{\overline{#1} + #1'}
%\newcommand{\inbr}[1]{\left( #1 \right)}
%\newcommand{\ol}[1]{\overline{#1}}
%\newcommand{\timesES}[2]{\epsilon_{jki} #1_j #2_k \unitvec{e}_i}
%\newcommand{\oh}[0]{\frac{1}{2}}
%\newcommand{\todoil}[1]{\todo[color=red]{#1}}
%\newcommand{\todoilc}[2]{\todo[color=#1]{#2}}
%\newcommand{\derref}[1]{\footnote{see\fcolorbox{white}{gray!20}{Derivation \ref{#1}}}
%}
%\newcommand{\derrefs}[1]{\footnote{see\fcolorbox{white}{gray!20}{Derivations \cref{#1}}}
%}
%% \mdfdefinestyle{definition}
%% {linewidth=1,
%% backgroundcolor=yellow!40,
%% outerlinecolor=blue!70!black,
%% frametitlebackgroundcolor=gray!20,
%% % frametitlerule=true,
%% innertopmargin=\topskip,}
%% \mdtheorem[
%% style=definition]{definition}{Definition}
%% \mdfdefinestyle{codepiece}
%% {linewidth=15pt,
%% linecolor=black,
%% %frametitlerule=true,
%% backgroundcolor=red!1,
%% outerlinecolor=black,
%% frametitlebackgroundcolor=blue!10,
%% innertopmargin=\topskip,}
%% \mdtheorem[
%% style=codepiece]{codepiece}{Code}[chapter]
%% \mdfdefinestyle{function}
%% {%linewidth=15pt,
%% %linecolor=black,
%% %frametitlerule=true,
%% %backgroundcolor=red!1,
%% %outerlinecolor=black,
%% %frametitlebackgroundcolor=blue!10,
%% innertopmargin=-.5cm,}
%% \mdtheorem[
%% style=function]{function}{sub-routine}[section]
%% \mdfdefinestyle{derivation}
%% {linecolor=red,
%% outerlinewidth=2,
%% leftmargin=40mm,
%% rightmargin=40mm,
%% linewidth=2,
%% frametitlerule=true,
%% frametitlebackgroundcolor=gray!20,
%% innertopmargin=\topskip,roundcorner=10pt,}
%% \mdtheorem[
%% style=derivation]{derivation}{Derivation}
%% \mdfdefinestyle{eddy}
%% {linewidth=.5,
%% % align=center,
%% % backgroundcolor=yellow!40,
%% outerlinecolor=black,
%% frametitlebackgroundcolor=gray!20,
%% % frametitlerule=true,
%% innertopmargin=\topskip,}
%% \mdtheorem[
%% style=eddy]{eddy}{Vortex}[chapter]
%% \mdtheorem[
%% style=eddy]{turbu}{Turbulence}[chapter]

\graphicspath{ {./FIGS/} }
\usepackage{multimedia}
\begin{document}
\title{eddy bla}  
\author{NK}
\date{\today} 
\begin{frame}
\titlepage
\end{frame}

%\begin{frame}
%\frametitle{Table of contents}
%\tableofcontents
%\end{frame} 


%\section{Introduction}


%\begin{frame}
%\begin{figure}
	%\centering
	%\includegraphics[width=300pt,keepaspectratio=true]{png1024x/atl1024x500PHONged.png}
%\end{figure}
%\end{frame}

%\begin{frame}
%\begin{figure}
	%\centering
	%\includegraphics[width=300pt,keepaspectratio=true]{shrunks/scale_04.pdf}
%\end{figure}
%\end{frame}

%\section{Detection}

%\begin{frame}
%\begin{figure}
	%\centering
	%\includegraphics[height=220pt,keepaspectratio=true]{shrunks/slice1.pdf}
%\end{figure}
%\end{frame}

%\begin{frame}
%\begin{figure}
	%\centering
	%\includegraphics[height=220pt,keepaspectratio=true]{shrunks/slice2.pdf}
%\end{figure}
%\end{frame}

%\begin{frame}
%\begin{figure}
	%\centering
	%\includegraphics[height=220pt,keepaspectratio=true]{shrunks/slice3.pdf}
%\end{figure}
%\end{frame}

%\begin{frame}
%\begin{figure}
	%\centering
	%\includegraphics[height=220pt,keepaspectratio=true]{shrunks/slice4.pdf}
%\end{figure}
%\end{frame}

%\begin{frame}
%\begin{figure}
	%\centering
	%\includegraphics[height=220pt,keepaspectratio=true]{shrunks/sliceAll.pdf}
%\end{figure}
%\end{frame}


%\begin{frame}
%\begin{figure}
	%\centering
	%\includegraphics[height=220pt,keepaspectratio=true]{shrunks/Non-filtered_SSH.pdf}
%\end{figure}
%\end{frame}

%\begin{frame}
%\begin{figure}
	%\centering
	%\includegraphics[width=300pt,keepaspectratio=true]{shrunks/High-pass_filtered_SSH.pdf}
%\end{figure}
%\end{frame}

%\begin{frame}
%\begin{figure}
	%\centering
	%\includegraphics[height=220pt,keepaspectratio=true]{shrunks/ch.pdf}
%\end{figure}
%\end{frame}

%\begin{frame}
%\begin{figure}
	%\centering
	%\includegraphics[height=220pt,keepaspectratio=true]{shrunks/slice5.pdf}
%\end{figure}
%\end{frame}

%\begin{frame}
 %\frametitle{Isoperimetric Quotient (IQ)}
%\begin{figure}
	%\includegraphics[height=150pt,keepaspectratio=true]{shrunks/isoper.pdf}
%\end{figure}
%\end{frame}


%\begin{frame}
%\begin{figure}
	%\centering
	%\includegraphics[height=220pt,keepaspectratio=true]{shrunks/iq6dlMrIc.pdf}
%\end{figure}
%\end{frame}


%\begin{frame}
%\begin{figure}
	%\centering
	%\includegraphics[height=220pt,keepaspectratio=true]{shrunks/iq6toiq55.pdf}
%\end{figure}
%\end{frame}

%%% TRACKING
%\section{Tracking}

%%%effectively looking at zero om contours. fickle (similar to OW, which is even d_xx squared!). gladly pop and aviso product are rel. clean in terms of noise.
%%% note: dyn radius <=> std(gaussbell)
%%% f(x)=exp(- (2 r_dyn)^-2   x^2 )

%\begin{frame}
%\frametitle{$\partial_x \sim \vec{U}, \; \partial_{xx} \sim \vec{\omega}, \;  \left(\partial_{xx}\right)^2 \sim OW \sim \enstro$}
%\begin{figure}
	%\centering
	%\includegraphics[height=220pt,keepaspectratio=true]{prof1.pdf}
%\end{figure}
%\end{frame}

%\begin{frame}
%\begin{figure}
	%\centering
	%\includegraphics[height=220pt,keepaspectratio=true]{prof2.pdf}
%\end{figure}
%\end{frame}

%\begin{frame}
 %\frametitle{$ y(x)=\sqrt{e} a \exp{\left(- x^2/2\sigma^2\right)}$}
%\begin{figure}
	%\centering
	%\includegraphics[height=220pt,keepaspectratio=true]{prof3.pdf}
%\end{figure}
%\end{frame}

%\begin{frame}
 %\frametitle{$ y(x)=\sqrt{e} a \exp{\left(- x^2/2\sigma^2\right)}$}
%\begin{figure}
	%\centering
	%\includegraphics[height=220pt,keepaspectratio=true]{prof3b.pdf}
%\end{figure}
%\end{frame}

%\begin{frame}
%\begin{figure}
%\centering
	%\includegraphics[width=300pt,keepaspectratio=true]{prof3bPsi.pdf}
%\end{figure}
%\end{frame}

%\begin{frame}
%% NOTE: 14day dt
%\begin{figure}
	%\centering
	%\includegraphics[height=250pt,keepaspectratio=true]{shrunks/scatRadOLrComp.pdf}
%\end{figure}
%\end{frame}

%\begin{frame}
%\begin{figure}
	%\centering
	%\includegraphics[width=300pt,keepaspectratio=true]{shrunks/parAnalyCH.pdf}
%\end{figure}
%\end{frame}

%\begin{frame}
%\begin{figure}
	%\centering
	%\includegraphics[width=300pt,keepaspectratio=true]{shrunks/parAnalyIQ2.pdf}
%\end{figure}
%\end{frame}

%\begin{frame}
%\begin{figure}
	%\centering
	%\includegraphics[width=300pt,keepaspectratio=true]{shrunks/parAnalyIQ6.pdf}
%\end{figure}
%\end{frame}

%\begin{frame}
%\begin{figure}
	%\centering
	%\includegraphics[width=300pt,keepaspectratio=true]{crpd_scat1_shrunk2prepress.pdf}
%\end{figure}
%\end{frame}

%\begin{frame}
%\begin{figure}
	%\centering
	%\includegraphics[width=300pt,keepaspectratio=true]{deflectsCycs_shrunk2prepress.pdf}
%\end{figure}
%\end{frame}

%\begin{frame}
 %\frametitle{Isoperimetric Quotient (IQ)}
%\begin{figure}
	%\centering
	%\includegraphics[width=300pt,keepaspectratio=true]{MapVelCycs.pdf}
%\end{figure}
%\end{frame}


\begin{frame}
	\frametitle{Okubo-Weiss}
	\centering
	\begin{minipage}[T]{1\textwidth}
	\begin{figure}
		\includegraphics[width=200pt,keepaspectratio=true]{OWpatch2.pdf}
	\end{figure}		
	\end{minipage}
\vfill
	\begin{minipage}[T]{1\textwidth}
		use eigenvalues of 2d deformation tensor to detect vortex: 
		%\begin{align*}
			%%det(\lambda\vec{I} \vec{\Lambda} - \vec{T})=0
			%%&=
			%%(\lambda-u_{x})(\lambda+u_{x}) - v_{x}u_{y} \\
			%%%&=
			%%%\lambda^{2}-u_{x}^{2} - v_{x}u_{y}\\ 
			%\lambda
			%&=
			%\pm \sqrt{u_{x}^{2} + v_{x}u_{y}}
		%\end{align*}
\begin{align*}
& det(\lambda\vec{I}- \vec{\grad \vec{u}}) =0\\
&\lambda^{2} =OW/2=\tikzmarkin[fill=yellow]{first eq}u_{x}^{2} + v_{x}u_{y}\tikzmarkend{first eq} \\
%&=-(\vort{u})^{2} + (\div{\vec{u}})^{2} + (\grad_{\updownarrow} \cdot \vec{u})^{2}	+ (\underaccent{90^{\circ}}{\grad} \cdot\vec{u})^{2}	\\	&=-vorticity^{2}+divergence^{2}+shear^{2}+stretching^{2}\\& = OW/4
\end{align*}		
\end{minipage}
\end{frame}


%\begin{frame}
%\begin{center}
	%\centering
%\href{run:/usr/bin/mplayer -fs MOViq55RoL4MadShort.avi}{
%\includegraphics[scale=0.25]
%{MOViq55RoL4MadShort.jpeg}}
%\end{center}
%\end{frame}


%%\begin{frame}
%% \frametitle{pictures in latex beamer class}
%\begin{figure}
	%\centering
%% 	\includegraphics[width=300pt,keepaspectratio=true]{GS.pdf}
	%\includegraphics[height=220pt,keepaspectratio=true]{scale_04.pdf}
%\end{figure}
%\end{frame}


%\begin{frame}
 %\frametitle{Thresholds}
%\begin{figure}
	%\centering
%% 	\includegraphics[width=300pt,keepaspectratio=true]{GS.pdf}
	%\includegraphics[height=220pt,keepaspectratio=true]{scat1.pdf}
%\end{figure}
%\end{frame}

%\begin{frame}
%%  \frametitle{Thresholds}
%\begin{figure}
	%\centering
%% 	\includegraphics[width=120mm,keepaspectratio=true]{GS.pdf}
	%\includegraphics[height=90mm,keepaspectratio=true]{CRPD_deflectsCycs.pdf}
%\end{figure}
%\end{frame}
\end{document}




\begin{frame}	
\begin{align*}
	\left| \lambda \vec{I} - \grad \vec{u} \right|
	&=
\begin{vmatrix}
 \lambda - u_{x} & u_{y} \\
 v_{x} & \lambda - v_{y}
 \end{vmatrix}	\\
 &=
 \left(  	\lambda - u_{x} \right)  \left( 	\lambda - v_{y} \right){}
 - v_{x} u_{y}\\
 \lambda^{2}&=
 	  u_{x}^{2}
 + v_{x} u_{y}
\end{align*}

\begin{align*}
	\left| \lambda \vec{I} - \vec{\Omega}  \right|
	&=
\frac{1}{2}\begin{vmatrix}
 \lambda & u_{y}-v_{x} \\
 -u_{y}+v_{x} & \lambda 
 \end{vmatrix}	\\
 2\lambda_{\Omega}^{2}
 &=
 u_{y}^{2}-2u_{y}v_{x}+v_{x}^{2}
\end{align*}

\begin{align*}
	\left| \lambda \vec{I} - \vec{E}  \right|
	&=
\frac{1}{2}\begin{vmatrix}
 \lambda -2u_{x} & u_{y}+v_{x} \\
 u_{y}+v_{x} & \lambda -2v_{y}
 \end{vmatrix}	\\
 %&(\lambda -2u_{x} ) (\lambda +2u_{x})
 %-u_{y}^{2}-2u_{y}v_{x}-v_{x}^{2}  
 %\\
  2\lambda_{E}^{2}
 &=4u_{x}^{2}
 +u_{y}^{2}
 + 2u_{y}v_{x}
 +v_{x}^{2}  
\end{align*}
\begin{align*}
	-\lambda_{\Omega}^{2} + \lambda_{E}^{2}
	&=
	 2u_{y}v_{x}
	 +2u_{x}^{2}
\end{align*}



\end{frame}



\begin{frame}
\begin{align*}
&=-(v_{x}-u_{y})^{2}
+(u_{x}+v_{y})^{2}
+(v_{x}+u_{y})^{2}
+(u_{x}-v_{y})^{2}\\
&=
-(
v_{x}^{2}
-2v_{x}u_{y}
+u_{y}^{2}
)
+0+
(
v_{x}^{2}
+2v_{x}u_{y}
+u_{y}^{2}
)
+(
u_{x}^{2}
-2u_{x}v_{y}
+v_{y}^{2}
)\\
&=
4v_{x}u_{y}
+(
u_{x}^{2}
-2u_{x}v_{y}
+v_{y}^{2}
)\\
&=
4v_{x}u_{y}
+
4u_{x}^{2}
\\
\end{align*} 
\end{frame}

